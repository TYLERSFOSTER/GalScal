\documentclass[letterpaper,11pt, reqno]{amsart}

\usepackage{amsfonts, amsthm, amssymb, amsmath, stmaryrd}
\usepackage{bbm}
\usepackage{mathrsfs,array}
\usepackage{eucal,fullpage,times,color,enumerate,accents,mathtools}
\usepackage{url}
\usepackage{scalerel,stackengine}
\usepackage{dotlessj}
\usepackage{cancel}
\usepackage{tikz}
\usepackage[all]{xy}
\usepackage{enumitem}
\usepackage{soul}
\usepackage{xcolor}
\usepackage{etoolbox}
\usepackage{pifont}
\usetikzlibrary{fadings}
\usetikzlibrary{calc}
\tikzfading[name=fade out,
  inner color=transparent!0, outer color=transparent!100]
\tikzfading[name=fade right,
  left color=transparent!0, right color=transparent!100]
\tikzfading[name=fade left,
  right color=transparent!0, left color=transparent!100]
\tikzfading[name=fade mid,
  left color = transparent!100, right color = transparent!100, middle color=transparent!0]
\usepackage{ifthen}
\tikzset{
  laser beam action/.style={
    line width=\pgflinewidth+1.4pt,draw opacity=.095,draw=#1,
  },
  laser beam recurs/.code 2 args={%
    \pgfmathtruncatemacro{\level}{#1-1}%
    \ifthenelse{\equal{\level}{0}}%
    {\tikzset{preaction={laser beam action=#2}}}%
    {\tikzset{preaction={laser beam action=#2,laser beam recurs={\level}{#2}}}}
  },
  laser beam/.style={preaction={laser beam recurs={20}{#1}},draw opacity=1,draw=#1},
}
\usepackage{graphics}
\usepackage{graphicx}
\usepackage[export]{adjustbox}
\usepackage[curve]{xypic}
\usepackage{bm}
\usetikzlibrary{calc}
\usepackage[font=small,labelfont=bf]{caption}
\usepackage{stackengine,scalerel,graphicx}
\savestack\UAtextstyle{\stackon[-2.7pt]{$\rule[2.3pt]{4pt}{.35pt}$}{\scalebox{-1}{$U$}}}
\def\UA{\scalerel*{\UAtextstyle}{X}}

\definecolor{navy}{rgb}{0,0,.65}

%This reverse-links the references in the paper. Useful for large papers.
\usepackage[colorlinks]{hyperref}
\hypersetup{colorlinks=true,urlcolor=teal,linkcolor=navy,citecolor=navy}

\makeatletter
\def\@tocline#1#2#3#4#5#6#7{\relax
  \ifnum #1>\c@tocdepth % then omit
  \else
    \par \addpenalty\@secpenalty\addvspace{#2}%
    \begingroup \hyphenpenalty\@M
    \@ifempty{#4}{%
      \@tempdima\csname r@tocindent\number#1\endcsname\relax
    }{%
      \@tempdima#4\relax
    }%
    \parindent\z@ \leftskip#3\relax \advance\leftskip\@tempdima\relax
    \rightskip\@pnumwidth plus4em \parfillskip-\@pnumwidth
    #5\leavevmode\hskip-\@tempdima
      \ifcase #1
       \or\or \hskip 1em \or \hskip 2em \else \hskip 3em \fi%
      #6\nobreak\relax
    \dotfill\hbox to\@pnumwidth{\@tocpagenum{#7}}\par
    \nobreak
    \endgroup
  \fi}
\makeatother

\renewcommand{\familydefault}{ppl}
\setlength{\marginparwidth}{1in}
\setlength{\marginparsep}{0in}
\setlength{\marginparpush}{0.1in}
\setlength{\topmargin}{0in}
\setlength{\headheight}{0pt}
\setlength{\headsep}{0pt}
\setlength{\footskip}{.3in}
\setlength{\textheight}{9.0in}
\setlength{\textwidth}{6.25in}
\setlength{\parskip}{0pt}

%\newtheorem{theorem}{Theorem}[section]
\newtheorem{composition}{Composition}[]
\renewcommand*{\thecomposition}{\Alph{composition}}
\newtheorem{theorem}{Theorem}[subsection]
\newtheorem{monodromy theorem}{Monodromy Theorem}[subsection]
\newtheorem{corollary}[theorem]{Corollary}
\newtheorem{claim}[theorem]{Claim}
\newtheorem{lemma}[theorem]{Lemma}
\newtheorem{proposition}[theorem]{Proposition}
\newtheorem{definition}[theorem]{Definition}
\newtheorem{warning}[theorem]{Warning}
\newtheorem{research objectives}{Research objectives}[subsection]
\newtheorem{questions}{Question}
\newtheorem{question}[theorem]{Question}
\newtheorem{research question}[theorem]{Research questions}
\newtheorem{answer}[theorem]{Answer}
\newtheorem{aside question}[theorem]{Aside question}
\newtheorem{exercise}[theorem]{Exercise}
\newtheorem{sketch}[theorem]{Sketch}
\newtheorem{problem}[theorem]{Problem}
\newtheorem{conjecture}[theorem]{Conjecture}
\newtheorem{assumption}[theorem]{Assumption}
\newtheorem{construction}[theorem]{Construction}
\newtheorem{example}[theorem]{Example}
\newtheorem{quasi-theorem}[theorem]{Quasi-Theorem}
\newtheorem{prop/def}[theorem]{Proposition/Definition}
\newtheorem{blank remark}[theorem]{}
\newtheorem{ssubsection}[theorem]{}
\newtheorem{terminology and comment}[theorem]{Terminology and comment}
\newtheorem{fact}[theorem]{Fact}
\newtheorem{computation}[theorem]{Computation}
\newtheorem{observation}[theorem]{Observation}
\newtheorem{setup}[theorem]{Setup}
\newtheorem{purity hypothesis}[theorem]{Purity hypothesis}
\newtheorem{corollary of the purity hypothesis}[theorem]{Corollary of the purity hypothesis}

%Theorem indexed with letters A, B, C, ...
\newtheorem{Th}{Theorem}[]
\renewcommand*{\theTh}{\Alph{Th}}

\newtheorem{rem1}[theorem]{Remark}
\newenvironment{remark}{\begin{rem1}\em}{\end{rem1}}

\newtheorem{not1}[theorem]{Notation}
\newenvironment{notation}{\begin{not1}\em}{\end{not1}}

%% Math Blackboard
\newcommand{\A}{{\mathbb{A}}}           
\newcommand{\CC} {{\mathbb C}}       
\newcommand{\DD} {{\mathbb D}}
\newcommand{\EE}{\mathbb{E}}
\newcommand{\GG}{\mathbb{G}}
\newcommand{\LL}{\mathbb{L}}
\newcommand{\NN} {{\mathbb N}}		
\newcommand{\PP}{\mathbb{P}}         
\newcommand{\QQ} {{\mathbb Q}}		
\newcommand{\RR} {{\mathbb R}}		
\newcommand{\Circ} {{\mathbb S}}		
\newcommand{\ZZ} {{\mathbb Z}}		
\newcommand{\TT} {{\mathbb T}}	
\newcommand{\FF}{{\mathbb F}}

%%Presuperscript
\def\presuper#1#2%
  {\mathop{}%
   \mathopen{\vphantom{#2}}^{#1}%
   \kern-\scriptspace%
   #2}
	
\newcommand{\BC}{\text{BC}}

\DeclareMathOperator{\Aut}{Aut}
\DeclareMathOperator{\Gal}{Gal}
\DeclareMathOperator{\Circpec}{Spec_{\ \!}}
\DeclareMathOperator{\Split}{split}
\DeclareMathOperator{\Div}{div}
\DeclareMathOperator{\ord}{ord_{\ \!}}

\newcommand{\model}[1]{{\slantbox[.5]{$\mathcal{#1}$}\ }}

\newcommand{\lra}{{\longrightarrow}}
\DeclareMathOperator{\Def}{\overset{{}_{\text{def}}}{=}}

% slant box
\newsavebox{\foobox}
\newcommand{\slantbox}[2][.5]
  {%
    \mbox
      {%
        \sbox{\foobox}{#2}%
        \hskip\wd\foobox
        \pdfsave
        \pdfsetmatrix{1 0 #1 1}%
        \llap{\usebox{\foobox}}%
        \pdfrestore
      }%
  }

%%Prettier monomorphism and epimorphism arrows
\newcommand{\mono}{\!\xymatrix{{}\ar@{^{(}->}[r]&{}}\!}
\newcommand{\epi}{\!\xymatrix{{}\ar@{->>}[r]&{}}\!}
\newcommand{\rat}{\!\xymatrix{{}\ar@{-->}[r]&{}}\!}

%%Young diagram
\newcommand{\young}{\scalebox{.7}{$\pmb{\square\!\square}${\larger\larger $\pmb{\cdot\!\cdot\!\cdot}$}$\pmb{\square}$}}

%smaller subscript closed field
\newcommand{\lilF}{\mbox{{\smaller\smaller\smaller\smaller\smaller $\overline{\FF_{\!q}}$}}}

\newcommand{\iso}{\cong}
\newcommand{\disc}{\text{disc}}

% Tyler comments
\newcommand{\tyler}[1]{{\color{red} [#1\ \ \textemdash Tyler]}}

% Some slanted letters
\newcommand{\TP}{\slantbox[.3]{$\mathcal{TP}$}}

%Left action
\newcommand{\lact}{\ \raisebox{8pt}{\rotatebox{-90}{$\circlearrowright$}}\ }

%Left quotient
\newcommand{\lquot}[2]{\raisebox{-1.5pt}{$#1$}\big\backslash\raisebox{1.5pt}{$#2$}}

%Right quotient
\newcommand{\rquot}[2]{\raisebox{1.5pt}{$#1$}/\raisebox{-1.5pt}{$#2$}}

%importantmatrix
\newcommand{\MM}{\big(\begin{smallmatrix}0 & -1\\ 1 & -1\end{smallmatrix}\big)}

%compactified spec
\newcommand{\SpecZN}[1]{\overline{\text{Spec}_{\ \!}\ZZ}{}^{(#1)}}
\newcommand{\SpecZ}{\overline{\text{Spec}_{\ \!}\ZZ}}

%nice sep
\newcommand{\sep}{\textsf{sep}}

%nice res
\newcommand{\res}{\text{res}^{\eta}_{s}}

%nice sp
\newcommand{\spe}{\bold{Sp}}

%nice Sh
\newcommand{\Sh}{\bold{Sh}}

%nice et
\newcommand{\et}{\text{\'et}}

%nice M_g bar
\newcommand{\Mg}{{\ \ \overline{\!\!\mathscr{M}_{g}}}}
\newcommand{\Mgmbar}{\overline{M_{g}{\!\!}^{\text{{\smaller\smaller\smaller\smaller\smaller $\ (m)$}}}}}
\newcommand{\Mgm}{M_{g}{\!\!}^{\text{{\smaller\smaller\smaller\smaller\smaller $\ (m)$}}}}


\stackMath
\newcommand\reallywidehat[1]{%
\savestack{\tmpbox}{\stretchto{%
  \scaleto{%
    \scalerel*[\widthof{\ensuremath{#1}}]{\kern-.6pt\bigwedge\kern-.6pt}%
    {\rule[-\textheight/2]{1ex}{\textheight}}%WIDTH-LIMITED BIG WEDGE
  }{\textheight}% 
}{0.5ex}}%
\stackon[1pt]{#1}{\tmpbox}%
}

%changed footnote style
\renewcommand{\thefootnote}{[\arabic{footnote}]}

\numberwithin{equation}{theorem}

%
\newcounter{totfigures}

\providecommand\totfig{} 

\makeatletter
\AtEndDocument{%
  \addtocounter{totfigures}{\value{figure}}%
  \immediate\write\@mainaux{%
    \string\gdef\string\totfig{\number\value{totfigures}}%
  }%
}
\makeatother

\pretocmd{\chapter}{\addtocounter{totfigures}{\value{figure}}\setcounter{figure}{0}}{}{}



\usepackage{epigraph}
\setlength\epigraphwidth{.8\textwidth}
\setlength\epigraphrule{0pt}


%%%%%%%%%%%%%%%%%%%%%%%%%%%%%%%%%%%%%%
%%%%%%%%%%%%%%%%%%%%%%%%%%%%%%%%%%%%%%

\begin{document}

\title{{\smaller\smaller\smaller Notes on}\\ Galois actions on algebraically tempered scales}
\author{Tyler Foster}
%\address{
%Department of Mathematics, Florida State University,
%Tallahassee, FL 
%32306}
%\email{\href{mailto:tsfoster@.fsu.edu}{tsfoster@fsu.edu}}

\date{\today}

\maketitle
\setcounter{tocdepth}{2}

\tableofcontents

\begin{section}{Mathematical setting}

\begin{subsection}{Tonic frequency as a vector}
At the outset, we take a {\em frequency} to be some kind of unit that we can rescale, as in the operation
\begin{equation}\label{equation: rescale}
\lambda\ \text{Hz}\ \ \ \longmapsto\ \ \ 2\lambda\ \text{Hz},
\end{equation}
and that we can add, as in the operation
\begin{equation}\label{equation: add}
\lambda\ \text{Hz},\ \ \ \mu\ \text{Hz}\ \ \ \longmapsto\ \ \ \lambda+\mu\ \text{Hz}.
\end{equation}
The ability to rescale corresponds to the ability to change pitch, to create musical movement along intervals. The ability to add pitches together is perhaps a bit suspect from a musical perspective, but it does correspond to nonlinear distortion phenomenon that occur in music, called \href{https://en.wikipedia.org/wiki/Combination_tone}{{\em Tartini tones}}.\footnote{\ ``In a power chord, the ratio between the frequencies of the root and fifth are very close to the just interval 3:2. When played through distortion, the intermodulation leads to the production of partials closely related in frequency to the harmonics of the original two notes, producing a more coherent sound. The intermodulation makes the spectrum of the sound expand in both directions, and with enough distortion, a new fundamental frequency component appears an octave lower than the root note of the chord played without distortion, giving a richer, more bassy and more subjectively `powerful' sound than the undistorted signal. Even when played without distortion, the simple ratios between the harmonics in the notes of a power chord can give a stark and powerful sound, owing to the resultant tone (combination tone) effect.''\ \textemdash\ Wikipedia, {\em Power chord}}

The operations \eqref{equation: rescale} and \eqref{equation: add} make frequencies look like elements in vector spaces. We will treat them as such.

Fix a frequency
\begin{equation}\label{equation: tonic}
\lambda\ \text{Hz}
\end{equation}
once and for all. 

\begin{subsection}{Just intoned scales as $\QQ$-vector spaces}
This frequency \eqref{equation: tonic} will serve as a {\em generator} for various scales that we will construct. Initially, we want to be somewhat agnostic about what it means to ``generate a scale from a frequency $\lambda\ \text{Hz}$,'' as we will end up giving several incompatible interpretations of what it means to ``generate a scale.''
\end{subsection}

\begin{example}
{\bf Just intoned scale with tonic $\pmb{\lambda}\ \text{Hz}$.}
\normalfont
If we take $\lambda\ \text{Hz}$ to be our tonic, then the commonly encountered {\em just intervals over this tonic} occur at the frequencies
	$$
	2\lambda\ \text{Hz},\ \ \ \ \ \ 
	\tfrac{3}{2}\lambda\ \text{Hz},\ \ \ \ \ \ 
	\tfrac{4}{3}\lambda\ \text{Hz},\ \ \ \ \ \ 
	\tfrac{5}{4}\lambda\ \text{Hz},\ \ \ \ \ \ 
	\tfrac{6}{5}\lambda\ \text{Hz},\ \ \ \ \ \ \text{and}\ \ \ \ \ \ 
	\tfrac{9}{8}\lambda\ \text{Hz}.
	$$
These are all instances of multiples $\tfrac{m}{n}\lambda\ \text{Hz}$ by rational numbers $\tfrac{m}{n}\in\QQ$ with relatively small \href{https://en.wikipedia.org/wiki/Height_function#Naive_height}{{\em naive multiplicative height}}. The set of all $\QQ$-multiples of the frequency $\lambda\ \text{Hz}$ is the $\QQ$-vector space
	\begin{equation}\label{equation: just vector space}
	V\ \ =\ \ \QQ\lambda\ \text{Hz}.
	\end{equation}
If we think of ``a just-intoned scale with tonic $\lambda\ \text{Hz}$'' as any collection of rational multiples of $\lambda\ \text{Hz}$, then the vector space \eqref{equation: just vector space} is the smallest just-intoned scale containing $\lambda\ \text{Hz}$ and containing all just intervals over all of its pitches. We call \eqref{equation: just vector space} the {\em complete just scale generated by the pitch $\lambda\ \text{Hz}$.}
\end{example}
\end{subsection}


\begin{subsection}{Alternate perspective: Frequencies are vectors in a Lie algebra}
The real reason we use frequencies is 
	$$
	f(t)
	\ =\ 
	Ae^{i\ \!2\pi\ \!\lambda\ \!t}
	$$
[...]
	$$
	Ae^{i\ \!2\pi\ \!\lambda\ \!t}
	\ \ =\ \ 
	e^{\text{log}A+i\ \!2\pi\ \!\lambda\ \!t}
	$$
[...]
	$$
	\begin{array}{rcl}
	u & \!\!:=\!\! & \text{log}_{\ \!}A \\
	v & \!\!:=\!\! & 2\pi\ \!\lambda\ \!t
	\end{array}
	$$
[...]
	$$
	e^{u+iv}=e^{z}.
	$$
[...]
	$$
	e^{(-)}\ \!:\ \CC\ \epi\ \CC^{\times}
	$$
[...]
	$$
	\text{exp}\ \!:\ \mathfrak{g}\ \lra\ G
	$$
[...]
\begin{enumerate}[label={\bf\ \ \ \ \ \ $\blacksquare$}]
\item {\bf Main premise.} A scales is a vector space $V$ that comes with natural ``exponentiation'' map $\text{exp}:V\lra \CC^\times$ to 
\end{enumerate}
	
\end{subsection}

\end{section}

\begin{section}{The Galois group of the 12-ET scale}
[...]
	$$
	\mathbb{Q}\ \mono\ \mathbb{Q}(i)
	$$
[...]
	$$
	\mathbb{Q}(i)\ \overset{2:1}{\mono}\ \mathbb{Q}\big(i,\sqrt{2}\big)\ \overset{2:1}{\mono}\ \mathbb{Q}\big(i,\sqrt[4]{2}\big)
	$$
[...]
	$$
	\mathbb{Q}\ \overset{2:1}{\mono}\ \mathbb{Q}(\zeta_3)\ \overset{3:1}{\mono}\ \mathbb{Q}\big(\zeta_3,\sqrt[3]{2}\big)
	$$
[...]
	$$
	e^{\zeta_3}
	\ \ =\ \ 
	e^{-1/2}\ e^{i\sqrt{3}/2}
	$$
\end{section}

\begin{section}{The Galois group of the 5-ET scale}
[...]
	$$
	\QQ\ \overset{4:1}{\mono}\ \QQ(\zeta_5)\ \overset{5:1}{\mono}\ \QQ(\zeta_5, \sqrt[5]{2})
	$$
[...]
	$$
	\zeta_{5}
	\ \ =\ \ 
	\tfrac{\ -1\ \!+\ \!\sqrt{5}\ }{4}\ +\ i\ \tfrac{\ \sqrt{10\ \!+\ \!2\ \!\sqrt{5}}\ }{4}
	$$
[...]
	$$
	\zeta_5\ \ \longmapsto\ \ \zeta^2_5
	$$
[...]
	$$
	\begin{array}{ccccc}
	1 & \zeta_5 & \zeta^2_5 & \zeta^3_5 & \zeta^4_5 \\[-4pt]
	{\color{gray!75}\rotatebox{-90}{$\mapsto$}} & {\color{gray!75}\rotatebox{-90}{$\mapsto$}} & {\color{gray!75}\rotatebox{-90}{$\mapsto$}} & {\color{gray!75}\rotatebox{-90}{$\mapsto$}} & {\color{gray!75}\rotatebox{-90}{$\mapsto$}} \\[12pt]
	1 & \zeta^2_5 & \zeta^4_5 & \zeta_5 & \zeta^3_5 \\[-4pt]
	{\color{gray!75}\rotatebox{-90}{$\mapsto$}} & {\color{gray!75}\rotatebox{-90}{$\mapsto$}} & {\color{gray!75}\rotatebox{-90}{$\mapsto$}} & {\color{gray!75}\rotatebox{-90}{$\mapsto$}} & {\color{gray!75}\rotatebox{-90}{$\mapsto$}} \\[12pt]
	1 & \zeta^4_5 & \zeta^3_5 & \zeta^2_5 & \zeta_5 \\[-4pt]
	{\color{gray!75}\rotatebox{-90}{$\mapsto$}} & {\color{gray!75}\rotatebox{-90}{$\mapsto$}} & {\color{gray!75}\rotatebox{-90}{$\mapsto$}} & {\color{gray!75}\rotatebox{-90}{$\mapsto$}} & {\color{gray!75}\rotatebox{-90}{$\mapsto$}} \\[12pt]
	1 & \zeta^3_5 & \zeta_5 & \zeta^4_5 & \zeta^2_5 \\[-4pt]
	{\color{gray!75}\rotatebox{-90}{$\mapsto$}} & {\color{gray!75}\rotatebox{-90}{$\mapsto$}} & {\color{gray!75}\rotatebox{-90}{$\mapsto$}} & {\color{gray!75}\rotatebox{-90}{$\mapsto$}} & {\color{gray!75}\rotatebox{-90}{$\mapsto$}} \\[12pt]
	\pmb{1} & \pmb{\zeta_5} & \pmb{\zeta^2_5} & \pmb{\zeta^3_5} & \pmb{\zeta^4_5}
	\end{array}
	$$
[...]
	$$
	A
	$$
\end{section}

\begin{section}{Implementation proposal 1: Galois-orbits of complex embeddings}
Here we provide examples of how one might implement the above ideas.

\begin{subsection}{Absolute amplitude- and time-scales}
Given a audio signal $f(t)$, we can rescale it amplitude via
	$$
	f(t)
	\ \ \longmapsto\ \ 
	A\cdot f(t)
	$$
for any $A\in\RR_{\ge 0}$. We can also rescale its rate via
	$$
	f(t)
	\ \ \longmapsto\ \ 
	f(\rho\cdot t)
	$$
[...]
	$$
	e^{\text{log}\ \!A+u}e^{i2\pi v\rho t}
	$$
	
\end{subsection}

\end{section}

\begin{section}{Implementation proposal 2: Lifts of Tate's additive $\zeta$-function}

\begin{subsection}{$\mathfrak{p}$-Adic fractional parts in number fields}
Before we described $\mathfrak{p}$-adic fractional parts of elements in number fields, it 's worthwhile to spend some time reviewing the theory for elements of $\QQ$.

\begin{ssubsection}
{\bf Fractional parts in $\pmb{\QQ}$.}
\normalfont
Fix a nonzero element $a\in\QQ$. Put this element in reduced form, so that
	$$
	a
	\ \ =\ \ 
	\pm
	\frac{M}{N}
	\ \ \ 
	\text{for elements}
	\ \ M,N\in\ZZ_{>0}.
	$$
For each prime ideal $\mathfrak{p}=(p)$ in $\ZZ$, there exist units $u_{M,p}$ and $u_{N,p}$ in $\ZZ_p$ so that
	$$
	M\ \ =\ \ p^m\cdot u_{M\!,\ \!p}
	\ \ \ \ \ \ \text{and}\ \ \ \ \ \ \ 
	N\ \ =\ \ p^n\cdot u_{N\!,\ \!p}.
	$$f
Defining $u_{p}:=u_{M\!,\ \!p}/u_{M\!,\ \!p}$ and we have
	$$
	a
	\ \ =\ \ 
	p^{\text{ord}_{p\!}(a)}\cdot u_{p}.
	$$
Here $u_p$ is a unit in $\mathbb{Z}_p$, so we can write it as
	$$
	u_p
	\ \ =\ \ 
	a_0 +a_1\ \!p + a_2\ \!p^2 + a_3\ \!p^3 + \cdots.
	$$
Thus
	$$
	\scalebox{.92}{$
	a
	\ \ =\ \ 
	a_0\ p^{\text{ord}_{p\!}(a)} + a_1\ p^{\text{ord}_{p\!}(a)+1}+\cdots+a_{-1-\text{ord}_{p\!}(a)}\ p^{-1}+a_{-\text{ord}_{p\!}(a)}+a_{1-\text{ord}_{p\!}(a)}\ p+a_{2-\text{ord}_{p\!}(a)}\ p^2+\cdots
	$}
	$$
The terms with $p$-adic absolute value $\le1$, i.e., the terms with nonnegative power of $p$ in this decomposition, form the {\em $p$-adic intergral part of $a$}, denoted $\lfloor a\rfloor_p$. The terms with $p$-adic absolute value $\le1$, i.e., the terms with nonnegative power of $p$ in this decomposition, form the {\em $p$-adic fractional part of $a$}:
	$$
	\scalebox{.92}{$
	a
	\ \ =\ \ 
	\underset{p\text{-adic fractional part }a-\lfloor a\rfloor_p}{\underbrace{a_0\ p^{\text{ord}_{p\!}(a)} + a_1\ p^{\text{ord}_{p\!}(a)+1}+\cdots+a_{-1-\text{ord}_{p\!}(a)}\ p^{-1}}}+
	\underset{p\text{-adic integral part }\lfloor a\rfloor_{p}}{\underbrace{a_{-\text{ord}_{p\!}(a)}+a_{1-\text{ord}_{p\!}(a)}\ p+a_{2-\text{ord}_{p\!}(a)}\ p^2+\cdots}}
	$}
	$$
\end{ssubsection}


\begin{example}
{\bf $\pmb{p}$-Adic fractional parts of the element $\pmb{a=\tfrac{1}{6}\in\mathbb{Q}}$.}
\normalfont
Le
\begin{enumerate}
\item
	\vskip .2cm
{\bf Case: $\pmb{p=2}$.}
We have $6=1\!\cdot\!2+1\!\cdot\!2^2=2\cdot(1+1\cdot2)$. Here
	$$
	\begin{array}{cc}
		&
		\ \begin{array}{cccccccccccccc}
		1 &\!\!+\!\!& 1\!\cdot\!2 &\!\!+\!\!& 1\!\cdot\!2^2 &\!\!+\!\!& 1\!\cdot\!2^3 &\!\!+\!\!& 1\!\cdot\!2^4 &\!\!+\!\!& 1\!\cdot\!2^5 &\!\!+\!\!& \cdots
		\end{array}
		\\
		1\ +\ 1\!\cdot\!2
		& 
		\begin{array}{|c}
		\hline
		\ \begin{array}{cccccccccccccc}
		1 &\!\!+\!\!& 0\!\cdot\!2 &\!\!+\!\!& 0\!\cdot\!2^2 &\!\!+\!\!& 0\!\cdot\!2^3 &\!\!+\!\!& 0\!\cdot\!2^4 &\!\!+\!\!& 0\!\cdot\!2^5 &\!\!+\!\!& \cdots
		\end{array}
		\end{array}
		\\
		&
		\ \begin{array}{cccccccccccccc}
		1 &\!\!+\!\!& 1\!\cdot\!2 & \ \ \ \ \ \ \ \ \ \ \ \ \ \ \ \ \ \ \ \ \ \ \ \ \ \ \ \ \ \ \ \ \ \ \ \ \ \ \ \ \ \ \ \ \ \ \ \ \ \ \ \ \ \ \ \ \ \ \ \ \ \ \ \ \ \ \ \ \ \ \ \ \ \ 
		\end{array}
		\\
		&
		\ \begin{array}{cccccccccccccc}
		1\!\cdot\!2 &\!\!+\!\!& 1\!\cdot\!2^2 & \ \ \ \ \ \ \ \ \ \ \ \ \ \ \ \ \ \ \ \ \ \ \ \ \ \ \ \ \ \ \ \ \ \ \ \ \ \ \ \ \ \ \ \ \ \ \ \ 
		\end{array}
		\\
		&
		\ \begin{array}{cccccccccccccc}
		1\!\cdot\!2^2 &\!\!+\!\!& 1\!\cdot\!2^3 & \ \ \ \ \ \ \ \ \ \ \ \ \ \ \ \ \ \ 
		\end{array}
		\\[-4pt]
		&
		\ \begin{array}{cccccccccccccc}
		\ \ \ \ \ \ \ \ \ \ddots
		\end{array}
	\end{array}
	$$
In other words, we have
	$$
	\frac{1}{1+1\!\cdot\!2}
	\ \ =\ \ 
	1+2+2^2+2^3+\cdots.
	$$
We can then use this to write $\frac{1}{6}$ in a form that makes the $2$-adic integral and fractional parts apparent:
	$$
	\frac{1}{6}
	\ \ \ =
	\underset{\text{part }\frac{1}{6}-\lfloor\frac{1}{6}\rfloor}{\underset{\text{fractional}}{\underset{2\text{-adic}}{\underbrace{\ 2^{-1}\ }}}}\!\!\!\!\!+\!\underset{2\text{-adic integral part }\lfloor\frac{1}{6}\rfloor}{\underbrace{\ \ 1+2+2^2+\cdots}\ \ }.
	$$
Thus
	$
	\text{frac}_{2}\big(\tfrac{1}{6}\big)
	=
	\tfrac{1}{2}
	$.
	\vskip .2cm
\item 
{\bf Case: $\pmb{p=3}$.}
We have $6=2\!\cdot\!3$
	\vskip .2cm
\item 
{\bf Case: $\pmb{p=5}$.}
We have $6=1+1\!\cdot\!5$
	\vskip .2cm
\item 
{\bf Case: $\pmb{p\ge7}$.}
We have $6=6\!\cdot\!p^0$
A
	\vskip .2cm
\end{enumerate}

\end{example}

\end{subsection}

$$
\ZZ[T]\big/(T^2+1)
$$
[...]
$$
T^2+1\ \ =\ \ (T-2)(T-3)
$$


\end{section}

\vskip 1cm

\bibliographystyle{amsalpha}
\bibliography{fault}

\end{document}
